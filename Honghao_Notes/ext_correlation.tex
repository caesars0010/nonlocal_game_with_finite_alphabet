\documentclass[11pt,letterpaper]{article}

\makeatletter
\renewcommand\paragraph{\@startsection{paragraph}{4}{\z@}%
                                      {1ex \@plus1ex \@minus.2ex}%
                                      {-1em}%
                                      {\normalfont\normalsize\bfseries}}
\makeatother

%%%%%%%%%%%%%%%%%%%%%
%  P A C K A G E S  %
%%%%%%%%%%%%%%%%%%%%%

% Authors
\usepackage{authblk}

% Page margins
\usepackage[margin=1in]{geometry}

% Nicer math font
\usepackage{mathpazo}

% More fancy lists
\usepackage{enumerate}

% Microtype
\usepackage{microtype}

% TikZ
\usepackage{tikz}
%\usetikzlibrary{calc,shapes.geometric}
\usetikzlibrary{backgrounds,fit,decorations.pathreplacing,calc}

% Highlights
\usepackage{soul}


% Figure
\usepackage{float}

% Hypertext package
\usepackage[colorlinks = true]{hyperref}
% Title and authors
%\hypersetup{
%  pdftitle = {},
%  pdfauthor = {}
%}
% Color definitions
\definecolor{darkred}  {rgb}{0.5,0,0}
\definecolor{darkblue} {rgb}{0,0,0.5}
\definecolor{darkgreen}{rgb}{0,0.5,0}
% Color links
\hypersetup{
  urlcolor   = blue,         % color of external links
  linkcolor  = darkblue,     % color of internal links
  citecolor  = darkgreen,    % color of links to bibliography
  filecolor  = darkred       % color of file links
}

% AMS
\usepackage{amsmath,amssymb,amsfonts,amsthm,amstext}

%% Restating theorems
%\usepackage{thm-restate}

% Powerful macros
\usepackage{etoolbox}

% Fixes for amsmath
\usepackage{mathtools}
\mathtoolsset{centercolon}
\makeatletter
\protected\def\tikz@nonactivecolon{\ifmmode\mathrel{\mathop\ordinarycolon}\else:\fi}
\makeatother

% Daw boxes
\usepackage{tcolorbox}

% Code
\usepackage{algorithm}
%\usepackage{algorithmic}
\usepackage{algpseudocode}

% Clever references

\usepackage{cleveref}%[nameinlink]


\crefname{lemma}{Lemma}{Lemmas}
\crefname{proposition}{Proposition}{Propositions}
\crefname{definition}{Definition}{Definitions}
\crefname{theorem}{Theorem}{Theorems}
\crefname{conjecture}{Conjecture}{Conjectures}
\crefname{corollary}{Corollary}{Corollaries}
\crefname{claim}{Claim}{Claims}
\crefname{section}{Section}{Sections}
\crefname{appendix}{Appendix}{Appendices}
\crefname{figure}{Fig.}{Figs.}
\crefname{table}{Table}{Tables}


% \crefname{algorithm}{Algorithm}{Algorithms}

% IEEE tools
\usepackage[retainorgcmds]{IEEEtrantools}

% table of contents
\usepackage{tocloft}

% Table with multi-row
\usepackage{multirow}

% TikZ
\usepackage{tikz}	
\usetikzlibrary{backgrounds,fit,decorations.pathreplacing}

%%%%%%%%%%%%%%%%%%%%%%%%%
%  N E W C O M M A N D  %
%%%%%%%%%%%%%%%%%%%%%%%%%

% Standard quantum notation

\newcommand{\ket}[1]{|#1\rangle}
\newcommand{\bra}[1]{\langle#1|}
\newcommand{\braket}[2]{\langle#1|#2\rangle}
\newcommand{\ketbra}[2]{|#1\rangle\langle#2|}
\newcommand{\proj}[1]{|#1\rangle\langle#1|}

\newcommand{\x}{\otimes}
\newcommand{\xp}[1]{^{\otimes #1}}
\newcommand{\op}{\oplus}

\newcommand{\ct}{^{\dagger}}
\newcommand{\tp}{^\intercal}

% Linear algebra

%\newcommand{\1}{\mathbb{1}} % identity matrix
%\DeclareMathOperator{\Hom}{Hom}
\DeclareMathOperator{\End}{End}
%\DeclareMathOperator{\E}{\mathbb{E}}
\DeclareMathOperator{\Lin}{L} % all linear maps
\newcommand{\Mat}[1]{\mathrm{M}(#1)} % all matrices
%\newcommand{\Mat}[1]{\mathrm{M}_{#1}(\C)}

% Paired delimiters

\DeclarePairedDelimiter{\set}{\lbrace}{\rbrace}
\DeclarePairedDelimiter{\abs}{\lvert}{\rvert}
\DeclarePairedDelimiter{\norm}{\lVert}{\rVert}
\DeclarePairedDelimiter{\cnorm}{\lvert}{\rvert}
\DeclarePairedDelimiter{\of}{\lparen}{\rparen}
\DeclarePairedDelimiter{\sof}{\lbrack}{\rbrack}
\DeclarePairedDelimiter{\ip}{\langle}{\rangle}
\DeclarePairedDelimiter{\floor}{\lfloor}{\rfloor}

% Operators

\renewcommand{\Re}{\operatorname{Re}}
\renewcommand{\Im}{\operatorname{Im}}
\DeclareMathOperator{\vc}{vec}
\DeclareMathOperator{\spn}{span}
\DeclareMathOperator{\rank}{rank}
\DeclareMathOperator{\diag}{diag}
\DeclareMathOperator{\spec}{spec}
\DeclareMathOperator{\Tr}{Tr}
\DeclareMathOperator{\sgn}{sgn}
\DeclareMathOperator{\hook}{hook}
\DeclareMathOperator{\E}{\mathbb{E}}
\DeclareMathOperator{\supp}{supp}


% Sets

\newcommand{\C}{\mathbb{C}}
\newcommand{\R}{\mathbb{R}}
\newcommand{\N}{\mathbb{N}}
\newcommand{\Z}{\mathbb{Z}}
\newcommand{\calH}{\mathcal{H}}
\newcommand{\calX}{\mathcal{X}}
\newcommand{\calY}{\mathcal{Y}}
\newcommand{\calA}{\mathcal{A}}
\newcommand{\calB}{\mathcal{B}}

% Group
\newcommand{\Zd}{\Z_d^{\times}}


% Identity operator
\newcommand{\1}{\mathbb{1}}

% Pauli Group
\newcommand{\Pg}{\mathcal{P}}
\newcommand{\J}{\mathcal{J}}

% Special notation

\newcommand{\CHSH}{CHSH^{(d)}}
\newcommand{\MS}{MS}
\newcommand{\EXT}{EXT}
\newcommand{\LS}{LS}
\newcommand{\COMM}{COMM}
\newcommand{\EPR}[1]{\Sigma^{(#1)}}
\newcommand{\paulix}{\sigma_x}
\newcommand{\pauliz}{\sigma_z}
\newcommand{\tP}{\tilde{P}}
\newcommand{\tQ}{\tilde{Q}}
\newcommand{\tM}{\tilde{M}}
\newcommand{\tN}{\tilde{N}}
\newcommand{\tA}{\tilde{A}}
\newcommand{\tB}{\tilde{B}}
\newcommand{\tX}{\tilde{X}}
\newcommand{\tZ}{\tilde{Z}}
\newcommand{\tU}{\tilde{U}}
\newcommand{\tW}{\tilde{W}}
\newcommand{\tx}{\tilde{x}}
\newcommand{\tpsi}{\tilde{\psi}}
\newcommand{\tri}{\Delta}
\newcommand{\lB}{\overline{B}}
\newcommand{\dr}[1]{d^{(#1)}}
\newcommand{\nr}{n(r)}
\newcommand{\mr}{m(r)}
\newcommand{\ux}{\underline{x}}
\newcommand{\uc}{\underline{c}}
\newcommand{\ua}{\underline{a}}
\newcommand{\ub}{\underline{b}}
\newcommand{\fC}{\mathfrak{C}}
\newcommand{\ba}{\pmb{a}}
\newcommand{\bb}{\pmb{b}}
\newcommand{\bc}{\pmb{c}}
\newcommand{\bS}{\mathrm{S}}

% Probabilities
\newcommand{\pr}[2]{P(#1|#2)}
\newcommand{\pa}[2]{P_A(#1|#2)}
\newcommand{\pb}[2]{P_B(#1|#2)}
\newcommand{\tpr}[2]{\tilde{P}(#1|#2)}

% Bell Ineqaulities
\newcommand{\I}{\mathcal{I}}

% Square root of epsilon
\newcommand{\ep}{\epsilon}
\newcommand{\se}{\sqrt{\epsilon}}
\newcommand{\qe}{\epsilon^{1/4}}
\newcommand{\sd}{\sqrt{d}}
\newcommand{\sr}{\sqrt{r}}
\newcommand{\qr}{r^{1/4}}
\newcommand{\qd}{d^{1/4}}

% Approximately equatli
\newcommand{\appd}[1]{\simeq_{#1}}

% Comments
\def\carl#1{{\color{blue} #1}}
\newcommand{\hf}[1]{\textcolor{red}{#1}}
\newcommand{\hfc}[1]{\textcolor{red}{#1 -H.F.}}

%%%%%%%%%%%%%%%%%%%%%%%%%
%  N E W T H E O R E M  %
%%%%%%%%%%%%%%%%%%%%%%%%%

\newtheorem{theorem}{Theorem}[section]
\newtheorem{lemma}[theorem]{Lemma}
\newtheorem{proposition}[theorem]{Proposition}
\newtheorem{definition}[theorem]{Definition}
\newtheorem{corollary}[theorem]{Corollary}
\newtheorem{conjecture}[theorem]{Conjecture}
\newtheorem{claim}[theorem]{Claim}
\newtheorem*{conjecture*}{Conjecture}
\newtheorem*{problem}{Problem}
\newtheorem*{example}{Example}

\theoremstyle{definition}
\newtheorem*{remark}{Remark}



%%%%%%%%%%%%%%%%
%   Document   %
%%%%%%%%%%%%%%%%

\begin{document}

\title{Extending the correlation}

\author[1]{Honghao Fu}


\renewcommand\Affilfont{\itshape\small}



\affil[1]{Department of Computer Science, Institute for Advanced Computer Studies, and Joint Center for Quantum \break Information and Computer Science, University of Maryland, College Park, MD 20742, USA}

\maketitle

In this report, we try to extend the definition of $\fC(d,r)$ to non-prime $d$.
That is, we would like to consider $(r,d)$ such that $r$ generates a cyclic subgroup
of $\Zd$. We assume that 
\begin{align*}
   r^k \equiv 1 \pmod{d},
\end{align*}
and $k < d/2$.
For example, we can pick $r = 2$ and $d = 2^k - 1$.

The game $\LS(r)$ is the same as before.
We would like to force Alice and Bob to use operators
\begin{align*}
    &U_A = U_B = \ketbra{x_{r^{k-1}}}{x_{r^0}}
    + \ketbra{x_{-r^{k-1}}}{x_{-r^0}}+
    \sum_{j=1}^{k-1} \left(\ketbra{x_{r^{j-1}}}{x_{r^j}} + \ketbra{x_{-r^{j-1}}}{x_{-r^j}}\right)\\
    &O_A = O_B = \sum_{j=0}^{k-1} \left(\omega_d^{r^j}\ketbra{x_{r^j}}{x_{r^j}} + \omega_d^{-r^j} \ketbra{x_{-r^j}}{x_{-r^j}}\right)
\end{align*}
Note that $O_A^d = O_B^d = \1$ and $U_A^k = U_B^k = \1$.
The operator $O_A$ can be split into two binary observables
\begin{align*}
    &M_1 = \sum_{j=0}^{k-1} \left( \omega_d^{r^j}\ketbra{x_{r^j}}{x_{-r^j}} +\omega_d^{-r^j} \ketbra{x_{-r^j}}{x_{r^j}}\right) \\
    &M_2 = \sum_{j=0}^{k-1} \left( \ketbra{x_{r^j}}{x_{-r^j}} + \ketbra{x_{-r^j}}{x_{r^j}}\right),
\end{align*}
and similarly on Bob's side.
Let the eigendecomposition of $U$ be 
\begin{align*}
    U = \sum_{j=0}^{k-1} \omega_k^j \left(\ketbra{u_j}{u_j}
    + \ketbra{u_{-j}}{u_{-j}}\right),
\end{align*}
then $U_A := U$ can be decomposed into
\begin{align*}
    &M_3 = \ketbra{u_0}{u_0} + 
    \ketbra{u_{-0}}{u_{-0}} +
    \sum_{j=1}^{(k-1)/2} \omega_k^j\left(\ketbra{u_j}{u_{k-j}} +  \ketbra{u_{-j}}{u_{j-k}}\right)
    +\omega_k^{-j}\left(\ketbra{u_{k-j}}{u_{j}} + \ketbra{u_{j-k}}{u_{-j}}\right) \\
    &M_4 = \ketbra{u_0}{u_0} + 
    \ketbra{u_{-0}}{u_{-0}} +
    \sum_{j=1}^{(k-1)/2} \ketbra{u_j}{u_{k-j}} +  
    \ketbra{u_{k-j}}{u_{j}} + 
    \ketbra{u_{-j}}{u_{j-k}} +  
    \ketbra{u_{j-k}}{u_{-j}}
\end{align*}
Note that here we assume $k$ is odd.
In the basis $\{ \ket{x_{r^j}}, \ket{x_{-r^j}}\}_j$, the form of $U$ is
\begin{align}
    U = \sum_{j=0}^{k-1}\left(\ketbra{x_{r^{j-1}}}{x_{r^j}}+\ketbra{x_{-r^{j-1}}}{x_{-r^j}}\right).
\end{align}
The shared state is
\begin{align*}
    \ket{\psi} = \frac{1}{\sqrt{2k}} \sum_{j=0}^{k-1}
    \left(\ket{x_{r^j}}\ket{x_{-r^j}} + \ket{x_{-r^j}}\ket{x_{r^j}}\right).
\end{align*}

In the extended weighted CHSH test, the special subspace is
$V = \spn\{ \ket{x_1}, \ket{x_{-1}}\} =
\spn\{\ket{1}, \ket{-1}\}$.
The computational basis $\{ \ket{r^j}, \ket{-r^j} \}_{j=0}^{k-1}$ are related to the basis
$
\{ \ket{x_{r^j}}, \ket{x_{-r^j}} \}_{j=0}^{k-1}
$
by
\begin{align*}
    &\ket{r^j} = -\frac{1}{\sqrt{2}}(\ket{x_{r^j}} + e^{-ir^j\pi/d}\ket{x_{-r^j}}) \\
    &\ket{-r^j} = \frac{i}{\sqrt{2}}(\ket{x_{r^j}} - e^{-ir^j\pi/d}\ket{x_{-r^j}}).
	%\ket{x_j} = \frac{-1}{\sqrt{2}}(\ket{j} + i\ket{d-j}), \\
	%\ket{x_{d-j}} = \frac{-e^{ij\pi/d}}{\sqrt{2}}(\ket{j} - i\ket{d-j})
\end{align*}
In the basis $\{ \ket{r^j}, \ket{-r^j} \}_j$, 
\begin{align}
    \ket{\psi} = \frac{1}{\sqrt{2k}} \sum_{j=0}^{k-1}
    e^{ir^j\pi/d}(\ket{r^j}\ket{r^j} + \ket{-r^j}\ket{-r^j}).
\end{align}

In the rest of the work, we follow the 
convention $\ket{1} \equiv \ket{r^0}$
and $\ket{-1} \equiv \ket{-r^0}$.
Alice and Bob get the same question from the set $\{0,\nr+1,\nr+2\}$, they always give the same answer.
Alice's and Bob's measurements for questions $0, \nr+1$ and $\nr+2$ are
\begin{align*}
	&\tP_0^0 = \tQ_0^0 = \Pi_V && \tP_0^2 = \tQ_0^2 = \Pi_{V^\perp} \\
	&\tP_{\nr+1}^0 = \tQ_{\nr+1}^0 = \ketbra{1}{1} &&
	\tP_{\nr+1}^1 = \tQ_{\nr+1}^1 = \ketbra{-1}{-1} \\
	&\tP_{\nr+2}^0 = \tQ_{\nr+2}^0 = \ketbra{1_{\times}}{1_{\times}} && \tP_{\nr+2}^1 = \tQ_{\nr+2}^1 = \ketbra{(-1)_{\times}}{(-1)_{\times}} \\
	&\tP_{\nr+1}^2 = \tQ_{\nr+1}^2 = \Pi_{V^\perp} &&
	 \tP_{\nr+2}^2 = \tQ_{\nr+2}^2 = \Pi_{V^\perp},
\end{align*}
where
\begin{align*}
    \ket{1_{\times}} = \frac{1}{\sqrt{2}}(\ket{1} + \ket{-1})
	&&\ket{(-1)_{\times}} = \frac{1}{\sqrt{2}}(\ket{1} - \ket{-1}).
\end{align*}
Intuitively, these questions force Alice and Bob to agree on the subspace $V$ and
on the basis of $V$ that they will measure.

The ideal strategy also ensures that 
when Alice gets $x \in \{1,2\}$ and Bob gets $y \in \{\nr+1, \nr+2\}$, 
their answers $a,b \in \{0,1\}$ should maximize $\I_{\cot(-\pi/d)}$ with Alice and Bob's roles flipped;
and when Alice gets $x \in \{\nr+1,\nr+2\}$ and Bob gets $y \in \{1, 2\}$,
their answers $a,b \in \{0,1\}$ should maximize $\I_{\cot(-\pi/d)}$.
This is the reason behind naming this test the extended weighted CHSH test.
It is also the reason behind having Alice and Bob agreeing on the subspace $V$ 
by using the other three questions.
Before introducing the measurements, 
we define some basis states of $V$
\begin{align*}
	&\ket{1_+} = \cos(-\frac{\pi}{2d})\ket{1} + \sin(-\frac{\pi}{2d})\ket{-1}
	&&\ket{(-1)_+} = \sin(-\frac{\pi}{2d})\ket{1} - \cos(-\frac{\pi}{2d})\ket{-1}\\
	&\ket{1_-} = \cos(-\frac{\pi}{2d})\ket{1} - \sin(-\frac{\pi}{2d})\ket{-1}
	&&\ket{(-1)_-} = \sin(-\frac{\pi}{2d})\ket{1} + \cos(-\frac{\pi}{2d})\ket{-1},
\end{align*}
so that $\{\ket{1_+}, \ket{(-1)}_+\}$ and $\{\ket{1_-}, \ket{(-1)_-}\}$ are two basis
of $V$.
When Alice and Bob gets question $1$ or $2$, what they measure on $V$ are
\begin{align*}
	&\tP_1^0 = \tQ_1^0 =  \ketbra{1_+}{1_+} && \tP_1^1 = \tQ_1^1 =  \ketbra{(-1)_+}{(-1)_+}\\
	&\tP_2^0 = \tQ_2^0 =  \ketbra{1_-}{1_-} && \tP_2^1 = \tQ_2^1 =  \ketbra{(-1)_-}{(-1)_-}.
\end{align*}
Notice that their measurements on $V$ follow the measurements to achieve $\I_{\cot(-\pi/d)}^{\max}$.
What they measure on $V^\perp$ is irrelevant for this test.

The generated correlation is given below.

\begin{table}[H]
\begin{center}
\begin{tabular}{|c|c||c|c|c|c|c|c|}
\hline
\multicolumn{2}{|c|}{} &
\multicolumn{3}{|c|}{$x=\nr+1$}&
\multicolumn{3}{|c|}{$x=\nr+2$} \\
\cline{3-8}
\multicolumn{2}{|c|}{} &
$a = 0$ & $a=1$ & $a=2$ &
$a = 0$ & $a=1$ & $a=2$\\
\hline
\hline
\multirow{2}{*}{$y = 1$} & $b=0$ & $\frac{\cos^2(\pi/2d)}{2k}$ & $\frac{\sin^2(\pi/2d)}{2k}$ & \small $\pr{2,0}{\nr+1,1}$ 
& $\frac{1-\sin(\pi/d)}{4k}$ & $\frac{1+\sin(\pi/d)}{4k}$ & \small  $\pr{2,0}{\nr+2,1}$ \\
\cline{2-8}
&$b=1$ & $\frac{\sin^2(\pi/2d)}{2k}$ & $\frac{\cos^2(\pi/2d)}{2k}$ & $\frac{k-1}{k}-\pr{2,0}{\nr+1,1}$ 
&  $\frac{1+\sin(\pi/d)}{4k}$ & $\frac{1-\sin(\pi/d)}{4k}$ & \small $\frac{k-1}{k} - \pr{2,0}{\nr+2,1}$  \\
\hline
\multirow{2}{*}{$y = 2$} & $b=0$ & $\frac{\cos^2(\pi/2d)}{2k}$ & $\frac{\sin^2(\pi/2d)}{2k}$ & \small $\pr{2,0}{\nr+1,2}$ & 
$ \frac{1+\sin(\pi/d)}{4k}$ & $ \frac{1-\sin(\pi/d)}{4k}$ & \small $\pr{2, 0}{\nr+2,2}$  \\
\cline{2-8}
&$b=1$ & $\frac{\sin^2(\pi/2d)}{2k}$ & $\frac{\cos^2(\pi/2d)}{2k}$ & \small $\frac{k-1}{k}-\pr{2,0}{\nr+1,2}$ &  
$ \frac{1-\sin(\pi/d)}{4k}$ & $ \frac{1+\sin(\pi/d)}{4k}$ & \small $\frac{k-1}{k}- \pr{2, 0}{\nr+2,2}$ \\
\hline
\end{tabular}
\end{center}
\caption{The correlation for $x \in \{n+1, n+2\}$ and $y \in \{1,2\}$.}
\label{tb:chsh}
\end{table}
We don't explicitly calculate the conditional probabilities of the form $\pr{\perp 0}{xy}$ for all possible $x$,$y$ 
because they are irrelevant.
Note that when $x \in \{1,2\}$ and $y \in \{\nr+1, \nr+2\}$, the correlation table is 
the transpose of Table~\ref{tb:chsh}, so we omit it here.

\begin{table}[H]
\begin{center}
\begin{tabular}{|c|c||c|c|c|c|c|c|c|c|}
\hline
\multicolumn{2}{|c|}{} &
\multicolumn{3}{|c|}{$x=\nr+1$}&
\multicolumn{3}{|c|}{$x=\nr+2$}&
\multicolumn{2}{|c|}{$x=0$}\\
\cline{3-10}
\multicolumn{2}{|c|}{} &
$a = 0$ & $a=1$ & $a=2$ &
$a = 0$ & $a=1$ & $a=2$ &
$a = 0$ & $a =2 $\\
\hline
\hline
\multirow{3}{*}{$y = \nr+1$} & $b=0$ & $\frac{1}{2k}$ & $0$ & 0 
& $\frac{1}{4k}$ & $\frac{1}{4k}$ & 0 & $\frac{1}{2k}$ & 0 \\
\cline{2-10}
&$b=1$ & 0 & $\frac{1}{2k}$ & $0$ 
&  $\frac{1}{4k}$ & $\frac{1}{4k}$ & 0 &$\frac{1}{2k}$ & 0 \\
\cline{2-10}
&$b=2$ & 0 & 0 & $\frac{k-1}{k}$ 
&  0 & 0 &  $\frac{k-1}{k} $ &0 & $\frac{k-1}{k}$ \\
\hline
\multirow{3}{*}{$y = \nr+2$} & $b=0$ & $\frac{1}{4k}$ & $\frac{1}{4k}$ & 0 
& $\frac{1}{2k}$ & $0$ & 0 & $\frac{1}{2k}$ & 0 \\
\cline{2-10}
&$b=1$ & $\frac{1}{4k}$ & $\frac{1}{4k}$ & $0$ 
&  0 & $\frac{1}{2k}$ & $0$ &$\frac{1}{2k}$ & 0 \\
\cline{2-10}
&$b=2$ & 0 & 0 & \small $\frac{k-1}{k}$ 
&  0 & 0 & \small $\frac{k-1}{k} $ &0 &\small $\frac{k-1}{k}$ \\
\hline
\multirow{2}{*}{$y = 0$} & $b=0$ & $\frac{1}{2k}$ & $\frac{1}{2k}$ & 0 
& $\frac{1}{2k}$ & $\frac{1}{2k}$ & 0 & $\frac{1}{k}$ & 0 \\
\cline{2-10}
&$b=2$ & 0 & 0 & $\frac{k-1}{k}$ 
&  0 & 0 & \small $\frac{k-1}{k} $ &0 & \small $\frac{k-1}{k}$ \\
\hline
\end{tabular}
\end{center}
\caption{The correlation for $x ,y\in\{\nr+1,\nr+2, 0\} $.}
\end{table}

\begin{table}[H]
\begin{center}
\begin{tabular}{|c|c||c|c|c|c|c|c|}
\hline
\multicolumn{2}{|c|}{} &
\multicolumn{6}{|c|}{$y=(\nr+1, f_0)$}\\
\cline{3-8}
\multicolumn{2}{|c|}{} &
$b = (0,0)$ & $b=(0,1)$ & 
$b = (1,0)$ & $b=(1,1)$ &
$b = (2,0)$ & $b=(2,1)$   \\
\hline
\hline
\multirow{3}{*}{$x = \nr+1$} & $a=0$ & $\frac{1}{2d-2}$ & $\frac{1}{2d-2}$ &  $0$
& $0$ & $0$ & $0$  \\
\cline{2-8}
&$a=1$ & $0$ & $0 $ & $\frac{1}{2d-2}$ 
&  $\frac{1}{2d-2}$ & $0$ & $0$  \\
\cline{2-8}
&$a=2$ & 0 & 0 & $0$ 
&  0 & $\frac{d-3}{2d-2}$ & $\frac{d-3}{2d-2} $  \\
\hline
\multirow{2}{*}{$x = f_0$} & $a=0$ & $\frac{1}{2d-2}$ & $0$ & $\frac{1}{2d-2}$ 
& $0$ & $\frac{d-3}{2d-2}$ & 0  \\
\cline{2-8}
&$a=1$ & $0$ & $\frac{1}{2d-2}$ & $0$ 
&  $\frac{1}{2d-2}$ & $0$ & $\frac{d-3}{2d-2}$  \\
\hline
\end{tabular}
\end{center}
\caption{The correlation for the commutation test between $f_0$ and $\nr+1$.}
\label{tbl:comm}
\end{table}



\end{document}