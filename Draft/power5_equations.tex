We have order-$2$ generators $\{x_i, w_i, y_i, j_i\}_{i=1}^{10} \cup \{f\} \cup \{g_{I_i}\}_{i=1}^8$,
where each $I_i$ represent a tuple of the form $(i,j,k)$ and it means that $x_ix_jx_i = x_k$.
We have 
\begin{align*}
	&I_1 = (2,1,5), 
	&&I_2 = (1,3,6), 
	&&I_3 = (2,6,7), 
	&&I_4 = (1,7,8), \\
	&I_5 = (4,1,9),
	&&I_6 = (3,8,9),
	&&I_7 = (4,2,10)
	&&I_8 = (3,2,10)
\end{align*}
and we collect them in a set $C = \{ I_i \}_{i=1}^8$. 
The linear relations involving the generators listed above are
\begin{align}
	\label{eq:xyz} &x_i y_i z_i = e  &&\text{for} \quad i =1\dots 10 \\
	\label{eq:xfw} &x_i f w_i = e   &&\text{for} \quad i =1\dots 10 \\
	&g_{I_1} y_1z_5 =e \\
	&g_{I_2} y_3z_6 =e \\
	&g_{I_3} y_6z_7 =e \\
	&g_{I_4} y_7z_8 =e \\
	&g_{I_5} y_1z_9 =e \\
	&g_{I_6} y_8z_9 = e\\
	&g_{I_7} y_2z_{10} =e \\
	&g_{I_8} y_2z_{10} = e
\end{align}

To introduce the following linear relations, we introduce another set of 
generators $\{\{y_{F_j,k}\}_{k=1}^6\}_{j=1}^{10}$.
For $j =1 \dots 10$ we have a group of linear relations of the form
\begin{align}
	f y_{F_j,1} y _{F_j,2} = e, && f y_{F_j,5} y_{F_j,6} = e \\
	y_j y_{F_j,2} y _{F_j,3} = e, && z_j y_{F_j,6} y_{F_j,7} = e \\
	y_{F_j,1} y_{F_j,4} y _{F_j,7} = e, && y_{F_j,3} y_{F_j,4} y_{F_j,5} = e,
\end{align}
We refer to each group of linear relations of the from above as $(F_j)$.

The last set of linear relations comes from each $K \in C$, 
so we introduce another set of generators $\{\{y_{K,k}\}_{k=1}^6\}_{K \in C}$.
We have a group of linear relations for each $K \in C$.
Suppose $K = (k_1,k_2,k_3)$, then the group of linear relations are
\begin{align}
	w_{k_1} y_{K,1} y _{K,2} = e, && w_{k_1} y_{K,5} y_{K,6} = e \\
	y_{k_2} y_{K,2} y _{K,3} = e, && z_{k_3} y_{K,6} y_{K,7} = e \\
	y_{K,1} y_{K,4} y _{K,7} = e, && y_{K,3} y_{K,4} y_{K,5} = e,
\end{align}
We refer to each group of linear relations as $(K)$.
In summary, we have $157$ binary generators, which are
\begin{align}
\{x_i, w_i, y_i, j_i\}_{i=1}^{10} \cup \{f\} \cup \{g_{I_i}\}_{i=1}^8\cup\{\{y_{F_j,k}\}_{k=1}^6\}_{j=1}^{10}
\cup \{\{y_{K,k}\}_{k=1}^6\}_{K \in C}.
\end{align}
There are $136$ linear relations.
Converting a linear relation to a linear equation is trivial, so we omit it here.

The derivation of $fy_jf=z_j$ from the group of linear relations $(F_j)$for $j = 1 \dots 10$
is the same as the previous case, so we omit it here.
We also omit the derivation of $x_{k_1} x_{k_2} x_{k_1} = x_{k_3}$, 
from the group of linear relations $(K)$,
for each $K \in C$.
In the end, we show the implication of the conjugacy relations in $C$ as follows 
\begin{align*}
	x_3 x_4 x_1x_2 x_4 x_3 =& x_3 (x_4 x_1x_4) (x_4x_2 x_4) x_3 \\
	=& (x_3 x_9 x_3) (x_3 x_{10} x_3)\\
	=& x_8 x_2\\
	=& (x_1x_7x_1) x_2\\
	=&x_1(x_2 x_6 x_2)x_1x_2\\
	=& x_1x_2(x_1x_5x_1)x_2x_1x_2\\
	=&x_1x_2x_1(x_2x_1x_2)x_1x_2x_1x_2
\end{align*}
where we used the fact that $x_i$'s are of order-$2$.
If we treat $x_3x_4$ as $u$ and $x_1x_2$ as $x$, then we have derived that $uxu^{-1} = x^5$.

