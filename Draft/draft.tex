\documentclass[11pt,letterpaper]{article}

\makeatletter
\renewcommand\paragraph{\@startsection{paragraph}{4}{\z@}%
                                      {1ex \@plus1ex \@minus.2ex}%
                                      {-1em}%
                                      {\normalfont\normalsize\bfseries}}
\makeatother

%%%%%%%%%%%%%%%%%%%%%
%  P A C K A G E S  %
%%%%%%%%%%%%%%%%%%%%%

% Authors
\usepackage{authblk}

% Page margins
\usepackage[margin=1in]{geometry}

% Nicer math font
\usepackage{mathpazo}

% More fancy lists
\usepackage{enumerate}

% Microtype
\usepackage{microtype}

% TikZ
\usepackage{tikz}
%\usetikzlibrary{calc,shapes.geometric}
\usetikzlibrary{backgrounds,fit,decorations.pathreplacing,calc}

% Highlights
\usepackage{soul}

% Young Tableaux
\usepackage{ytableau}

% Figure
\usepackage{float}

% Hypertext package
\usepackage[colorlinks = true]{hyperref}
% Title and authors
%\hypersetup{
%  pdftitle = {},
%  pdfauthor = {}
%}
% Color definitions
\definecolor{darkred}  {rgb}{0.5,0,0}
\definecolor{darkblue} {rgb}{0,0,0.5}
\definecolor{darkgreen}{rgb}{0,0.5,0}
% Color links
\hypersetup{
  urlcolor   = blue,         % color of external links
  linkcolor  = darkblue,     % color of internal links
  citecolor  = darkgreen,    % color of links to bibliography
  filecolor  = darkred       % color of file links
}

% AMS
\usepackage{amsmath,amssymb,amsfonts,amsthm,amstext}

%% Restating theorems
%\usepackage{thm-restate}

% Powerful macros
\usepackage{etoolbox}

% Fixes for amsmath
\usepackage{mathtools}
\mathtoolsset{centercolon}
\makeatletter
\protected\def\tikz@nonactivecolon{\ifmmode\mathrel{\mathop\ordinarycolon}\else:\fi}
\makeatother

% Daw boxes
\usepackage{tcolorbox}

% Code
\usepackage{algorithm}
%\usepackage{algorithmic}
\usepackage{algpseudocode}

% Clever references
\usepackage{cleveref}%[nameinlink]
\crefname{lemma}{Lemma}{Lemmas}
\crefname{proposition}{Proposition}{Propositions}
\crefname{definition}{Definition}{Definitions}
\crefname{theorem}{Theorem}{Theorems}
\crefname{conjecture}{Conjecture}{Conjectures}
\crefname{corollary}{Corollary}{Corollaries}
\crefname{claim}{Claim}{Claims}
\crefname{section}{Section}{Sections}
\crefname{appendix}{Appendix}{Appendices}
\crefname{figure}{Fig.}{Figs.}
\crefname{table}{Table}{Tables}
% \crefname{algorithm}{Algorithm}{Algorithms}

% IEEE tools
\usepackage[retainorgcmds]{IEEEtrantools}

% table of contents
\usepackage{tocloft}

%%%%%%%%%%%%%%%%%%%%%%%%%
%  N E W C O M M A N D  %
%%%%%%%%%%%%%%%%%%%%%%%%%

% Standard quantum notation

\newcommand{\ket}[1]{|#1\rangle}
\newcommand{\bra}[1]{\langle#1|}
\newcommand{\braket}[2]{\langle#1|#2\rangle}
\newcommand{\ketbra}[2]{|#1\rangle\langle#2|}
\newcommand{\proj}[1]{|#1\rangle\langle#1|}

\newcommand{\x}{\otimes}
\newcommand{\xp}[1]{^{\otimes #1}}
\newcommand{\op}{\oplus}

\newcommand{\ct}{^{\dagger}}
\newcommand{\tp}{^{\mathsf{T}}}

% Linear algebra

%\newcommand{\1}{\mathbb{1}} % identity matrix
%\DeclareMathOperator{\Hom}{Hom}
\DeclareMathOperator{\End}{End}
%\DeclareMathOperator{\E}{\mathbb{E}}
\DeclareMathOperator{\Lin}{L} % all linear maps
\newcommand{\Mat}[1]{\mathrm{M}(#1)} % all matrices
%\newcommand{\Mat}[1]{\mathrm{M}_{#1}(\C)}

% Paired delimiters

\DeclarePairedDelimiter{\set}{\lbrace}{\rbrace}
\DeclarePairedDelimiter{\abs}{\lvert}{\rvert}
\DeclarePairedDelimiter{\norm}{\lVert}{\rVert}
\DeclarePairedDelimiter{\of}{\lparen}{\rparen}
\DeclarePairedDelimiter{\sof}{\lbrack}{\rbrack}
\DeclarePairedDelimiter{\ip}{\langle}{\rangle}
\DeclarePairedDelimiter{\floor}{\lfloor}{\rfloor}

% Operators

\renewcommand{\Re}{\operatorname{Re}}
\renewcommand{\Im}{\operatorname{Im}}
\DeclareMathOperator{\vc}{vec}
\DeclareMathOperator{\spn}{span}
\DeclareMathOperator{\rank}{rank}
\DeclareMathOperator{\diag}{diag}
\DeclareMathOperator{\spec}{spec}
\DeclareMathOperator{\Tr}{Tr}
\DeclareMathOperator{\sgn}{sgn}
\DeclareMathOperator{\hook}{hook}
\DeclareMathOperator{\E}{\mathbb{E}}
\DeclareMathOperator{\supp}{supp}


% Sets

\newcommand{\C}{\mathbb{C}}
\newcommand{\R}{\mathbb{R}}
\newcommand{\N}{\mathbb{N}}
\newcommand{\Z}{\mathbb{Z}}
\newcommand{\calH}{\mathcal{H}}

% Identity operator
\newcommand{\1}{\mathbb{1}}

% Pauli Group
\newcommand{\Pg}{\mathcal{P}}
\newcommand{\J}{\mathcal{J}}

% Special notation

\newcommand{\CHSH}{CHSH^{(d)}}
\newcommand{\MS}{MS}
\newcommand{\SVT}{SVT}
\newcommand{\EPR}[1]{EPR^{(#1)}}
\newcommand{\paulix}[1]{\sigma_x^{(#1)}}
\newcommand{\pauliz}[1]{\sigma_z^{(#1)}}
\newcommand{\G}[1]{G^{(#1)}}
\newcommand{\LS}{LS}

% Probabilities
\newcommand{\pr}[2]{P(#1|#2)}
\newcommand{\pa}[2]{P_A(#1|#2)}
\newcommand{\pb}[2]{P_B(#1|#2)}

% Bell Ineqaulities
\newcommand{\I}{\mathcal{I}}



%%%%%%%%%%%%%%%%%%%%%%%%%
%  N E W T H E O R E M  %
%%%%%%%%%%%%%%%%%%%%%%%%%

\newtheorem{theorem}{Theorem}
\newtheorem{lemma}[theorem]{Lemma}
\newtheorem{proposition}[theorem]{Proposition}
\newtheorem{definition}[theorem]{Definition}
\newtheorem{corollary}[theorem]{Corollary}
\newtheorem{conjecture}[theorem]{Conjecture}
\newtheorem{claim}[theorem]{Claim}
\newtheorem*{conjecture*}{Conjecture}
\newtheorem*{problem}{Problem}
\newtheorem*{example}{Example}

\theoremstyle{definition}
\newtheorem*{remark}{Remark}



%%%%%%%%%%%%%%%%
%   Document   %
%%%%%%%%%%%%%%%%

\begin{document}

\title{Self-test EPR pair with constant alphabet}

\author[1]{Honghao Fu}
\author[1,2]{Carl Miller}

\renewcommand\Affilfont{\itshape\small}


\affil[1]{Department of Computer Science, Institute for Advanced Computer Studies, and Joint Center for Quantum \break Information and Computer Science, University of Maryland, College Park, MD 20742, USA}
\affil[2]{National Institute of Standards and Technology, 100 Bureau Dr., Gaithersbug, MD 20899, USA}
\maketitle
%========================================
\section{Preliminaries and notations}
%========================================
We use $[n]$ to denote the set $\{0,1 \dots n-1\}$.
We denote the group commutator of $A$ and $B$, i.e. $ABA^{-1}B^{-1}$, by $[A,B]$.

\textbf{The EPR pair.} Our goal is to self-test maximally entangled state of local dimension $d$, denoted by 
\begin{align}
\ket{\EPR{d}} = \frac{1}{\sqrt{d}} \sum_{i = 0}^{d-1} \ket{ii}.
\end{align}
The superscript $(d)$ is to stress the local dimension and we follow this convention through this paper.

\textbf{The Pauli operators.} 
We self-test $\ket{\EPR{d}}$ by verify that Alice and Bob has operators behave like 
the $d$-dimensional Pauli operators which are defined by
\begin{align}
	\paulix{d} = \sum_{i=0}^{d-1} \ketbra{i+1}{i} \quad \pauliz{d} = \sum_{i=0}^{d-1} \omega_d^i\ketbra{i}{i}
\end{align}
where $\omega_d = e^{i\pi/d}$ is the primitive $d$th root of unity and the addition is taken modulo $d$.

\textbf{The weighted CHSH inequality \cite{acin2012}.}
The first building-block of our result is a robust self-testing result based on the weighted CHSH inequality.
The weighted CHSH operator is defined as 
\begin{align}
	\label{eq:chsh_op}
	\I_\alpha = \alpha(A_0B_0+A_0B_1) + A_1B_0 - A_1B_1,
\end{align}
where $A_x,B_y$ for $x,y = 0,1$ are Binary observables on Hilbert space $\calH_A$ and $\calH_B$ respcetively.
If Alice and Bob share product state $\ket{\phi} = \ket{\phi_A} \x \ket{\phi_B}$, we have 
\begin{align}
	\bra{\phi} \I_\alpha \ket{\phi} \leq 2\alpha.
\end{align}
However, If they share entangled state $\ket{\psi}$, the maximal violation is 
\begin{align}
\label{eq:chsh_max}
\bra{\psi} \I_\alpha \ket{\psi} \leq 2\sqrt{1+\alpha^2}.
\end{align}
\begin{definition}[Ideal strategy for $\I_\alpha$]
	Define $\mu = \arctan(1/\alpha)$.
	The ideal strategy for weighted CHSH with parameter $\alpha$ (i.e. achieving maximal violation of \cref{eq:chsh_max})
	consists of the joint state $\ket{\EPR{2}}$ and observables $A_0 = \pauliz{2}$, $A_1 = \paulix{2}$,
	$B_0 = \cos(\mu) \pauliz{2} + \sin(\mu) \paulix{2}$ and $B_1 = \cos(\mu) \pauliz{2} - \sin(\mu) \paulix{2}$.
\end{definition}
We take an approach introduced in Ref.~\cite{bamps2015} and prove the following robust self-testing result.
\begin{theorem}
\label{thm:selftest}
	Suppose the quantum strategy $(\ket{\psi}, \{\tilde{A}_x\}_{x \in [2]}, \{\tilde{B}_y\}_{y \in [2]})$ satisfies that
	\begin{align}
		\bra{\psi} \I_\alpha \ket{\psi} \geq 2\sqrt{1+\alpha^2} - \epsilon
	\end{align}
	for some $\alpha$, then
	there exists a local isometry $\Phi = \Phi_A \x \Phi_B$ and an auxiliary state $\ket{aux}$  such that
	\begin{align}
		\| \Phi( \tilde{A}_x \x \tilde{B}_y \ket{\psi}) -\ket{aux} \x (A_x \x B_y) \ket{\EPR{2}}  \| = O(\sqrt{\epsilon})
	\end{align}
	for $x,y \in \{-1, 0, 1\}$ where the subscript $-1$ refers to the identity operator and where $A_x, B_y$ are from the 
	ideal strategy.
\end{theorem}
We defer the proof of \cref{thm:selftest} till Appendix~\ref{sec:selftest}.
After proving the robust the self-testing result, we take one step further and observed some interesting behaviour of the 
observable $\tilde{B}_0\tilde{B}_1$.
\begin{proposition}
\label{prop:2d-subspace}
	Suppose a quantum strategy $(\ket{\psi}, \{\tilde{A}_x\}_{x \in [2]}, \{\tilde{B}_y\}_{y \in [2]})$ achieves the maximal 
	value of  $\bra{\psi}\I_{-\cot(\pi/2d)} \ket{\psi}$, then there exists a $2$-dimensional Hilbert space which is spanned by eigenvectors of 
	$\tilde{B}_0\tilde{B}_1$ of eigenvalue $\omega_d$ and $\omega_d^{-1}$.
\end{proposition}
\begin{proof}[Proof of \cref{prop:2d-subspace}]
	By \cref{thm:selftest}, the condition that the strategy $(\ket{\psi}, \{\tilde{A}_x\}_{x \in [2]}, \{\tilde{B}_y\}_{y \in [2]})$ achieves the  
	maximal value of $\bra{\psi}  \I_\alpha \ket{\psi} $ implies that there exists state $\ket{u_0}, \ket{u_1}$ such that 
	\begin{align*}
	&(\tilde{B}_0 + \tilde{B}_1) \ket{u_0} = 2\cos(\pi/2d) \ket{u_0} \\
	&(\tilde{B}_0 + \tilde{B}_1) \ket{u_1} = -2\cos(\pi/2d) \ket{u_1}\\
	&(\tilde{B}_0 - \tilde{B}_1) \ket{u_0} = -2\sin(\pi/2d) \ket{u_1} \\
	&(\tilde{B}_0 - \tilde{B}_1) \ket{u_1} = -2\sin(\pi/2d) \ket{u_0}.
	\end{align*}
	It is straightforward to calculate that 
	\begin{align}
		\tilde{B}_0\tilde{B}_1 \ket{u_0} &= \cos(\pi/d) \ket{u_0} -\sin(\pi/d) \ket{u_1}\\
		\tilde{B}_0\tilde{B}_1\ket{u_1} &= \sin(\pi/d)\ket{u_0} + \cos(\pi/d) \ket{u_1}.
	\end{align}
	We can conclude that 
	\begin{align}
		&\tilde{B}_0\tilde{B}_1(\ket{u_0} + i\ket{u_1}) = e^{i \frac{\pi}{d}} (\ket{u_0} + i\ket{u_1})\\
		&\tilde{B}_0\tilde{B}_1(\ket{u_0} - i\ket{u_1}) = e^{-i \frac{\pi}{d}} (\ket{u_0} - i\ket{u_1}).
	\end{align}
\end{proof}

\textbf{Nonlocal game}. The two players of a nonlocal game are Alice and Bob. Each of them is requested
to give answer for a chosen question. We denote Alice's question set by $X$ and answer set by $A$. Similarly,
Bob's question set is denoted by $Y$ and his answer set is denoted by $B$. The nonlocal game also
comes with two functions: $\pi: X \times Y \rightarrow [0,1]$, which is the probability distribution over the questions,
and $V: A \times B \times X \times Y \rightarrow \R$, which is the scoring function. Such games are nonlocal
because Alice and Bob cannot communicate after getting their questions but they may share some strategy before 
the start of the game. Note that in the literature, the typical scoring function of a nonlocal game maps the input-output
pair to $\{0.1\}$ which corresponds to losing and winning. Allowing the score to be any real number is the key ingredient 
to our new nonlocal game. 

The quantum strategy of a game $G$ consists of projective measurements $\{\{A_x^a\}_a\}$ on Alice's side and 
$\{\{B_y^b\}_b\}_y$ on Bob's side, and a shared state $\ket{\psi}$. Then the behaviour of Alice and Bob is described 
by the conditional probability
\begin{align}
	\pr{ab}{xy} = \bra{\psi} A_x^a \x B_y^b \ket{\psi} \text{ for } (a,b,x,y) \in A \times B \times X \times Y,
\end{align}
where $(A_x^a)^2 = A_x^a = (A_x^a)^\dagger$ and $(B_y^b)^2 = B_y^b = (B_y^b)^\dagger$.
The \emph{value} of a strategy is given by
\begin{align}
	\omega(G,p)  = \sum_{a,b,x,y} \pi(x,y) \pr{ab}{xy} V(a,b,x,y).
\end{align} 

The main contribution of our work is the construction of a nonlocal game $\G{d}$ that can be used to self-test $\ket{\EPR{d}}$where $d$ is an arbitrary odd prime number. 
In fact, our nonlocal game self-tests $\paulix{d}$ and $\pauliz{d}$ too, just implicitly.
The nonlocal game is a linear system game with modifications. 
We introduce the definition of self-testing first and then the definition of linear system game, 
which come from Ref.~\cite{coladan2017, slofstra2017}.
\begin{definition}[Self-testing]
	We say that a nonlocal game self-tests a quantum state $\ket{\Psi}$ if 
	any quantum strategy $S$ that achieves the optimal quantum value uses a shared state equivalent up to local isometry to $\ket{\Psi}$.
\end{definition}
\begin{definition}[Linear system game]
 Let $Ax = b$ be an $m \times n$ linear system over $\Z_d$. The associated linear system game has two
 players Alice and Bob, where Alice is given a equation number $1 \leq x \leq m$ and replies with $a \in \Z_d^{\times n}$
 ,and Bob is given a variable $y$ and replies with an assignment $y \in \Z_d$. The winning condition is 
 \begin{align*}
 	a(y) &= b && \text{(Consistency)} \\
	\sum_{y = 1}^n A_{xy} a(y) &\equiv b(x) \pmod d. &&\text{(Constraint satisfaction)}
 \end{align*}
\end{definition}
The scoring function of linear system games always maps an input-output pair to $\{0,1\}$, 
so later when we say a quantum strategy wins a linear system game perfectly, we mean that 
$V(a,b,x,y) =0$ implies that $\pr{ab}{xy} = 0$.

The main tool to understand linear system game is through its solution group over $\Z_d$.
\begin{definition}[Solution group over $\Z_d$ \cite{coladan2017}]
	For the linear system game associated with $Ax = b$ over $\Z_d$, its solution group $\Gamma(A,b,\Z_d)$ has 
	one generator for each variable and one relation for each equation and relations enforcing that the variables in the same
	equation commutes. The set of local commutativity relations is denoted by $R_c$ and defined by
	\begin{align}
		R_c := \set{ [x_i, x_j] |  A_{li} \neq 0 \neq A_{lj} \text{ for some } 1 \leq l \leq m}.
	\end{align}
	The set of constraint satisfaction relations is denoted by $R_{eq}$ and defined by.
	\begin{align}
		R_{eq} := \set{ \J^{-b(l)} \Pi_{i=1}^n  x_i^{A_{li}} | 1\leq l\leq m}
	\end{align}
	Then the solution group has presentation 
	\begin{align}
	\Gamma(A,b,\Z_d) := \ip{ \{x_i\}_{i=1}^n \cup \{\J\} : R_c \cup R_{eq} \cup \{ (x_i)^d, \J^d| 1 \leq i \leq n\}}. 
	\end{align}
\end{definition}
Note that the relations $(x_i)^d$ and $\J^d$ ensure that we have solutions over $\Z_d$.
Then the operator solution of $\Gamma(A,b,\Z_d)$ is given below.
\begin{definition}[Operator solution]
	An operator solution for the linear system game associated with $Ax =b$ over $\Z_d$ is a unitary representation
	$\tau$ of $\Gamma(A,b,\Z_d)$ such that $\tau(\J) = \omega_d\1$. A conjugate operator solution is a unitary 
	representation mapping $\J$ to $\overline{\omega_d}\1$.
\end{definition}
What has been established in Ref.\cite{cleve2017,coladan2017} is that we can construct a perfect strategy of a
linear system game from its operator solution and vice versa.
%========================================
\section{Components of $\G{d}$}
%========================================
%----------------------------------------------------------------
\subsection{The linear system game}
%----------------------------------------------------------------
The main component of $\G{d}$ is a linear system game $\LS$, whose solution group is an embedding of the qudit Pauli group,
which is defined as
\begin{align}
	\Pg_d = \langle x, z, \J : x^d = z^d = \J^d = e, zxz^{-1}x^{-1} = \J, x\J x^{-1}\J^{-1}= z\J z^{-1}\J^{-1} = e\rangle. 
\end{align}
To construct $\LS$, we introduce new generators $u_x$ and $u_z$ to $\Pg_d$, and replace the relation $x^d = z^d = e$ 
by the following relations
\begin{align}
\label{eq:sim}
	u_x x u_x = x^2, \quad
	u_z z u_z = z^2.
\end{align}
Such relations give us the freedom to use the same solution group to embed different $\Pg_d$.

With constraints imposed by other components of the whole game, the two conditions above 
imply that the eigenvalues of $x$ and $z$ are $\{\omega_d^k = e^{ik\pi/d}\}_{k=1}^d$ with $d$ odd, prime and
$\Z/(d\Z)$ has primitive root $2$ and the eigenspaces for different eigenvalues are of the same dimension.
\hl{Later we will set the exponent to be $a \in \{2, 3, 5\}$ which is the primitive root of infinitely many prime numbers and
construct a nonlocal game that can self-test infinitely many maximally entangled state.}
\footnote{Figuring out what $a$ is will take us one step closer to resolving Artin's Conjecture\cite{murty1988}.}

Secondly, we drop the relation $\J^d = e$ and make the value of $\J$ determined by $x$ and $z$ in the 
relation $xzx^{-1}z^{-1} = \J$.

It can be easily checked that the qudit Pauli-x and Pauli-z operators of dimension $d$ satisfy the new relations 
above, where Pauli-x and Pauli-z operators are defined by
\begin{align}
	\sigma_x = \sum_{i=0}^{d-1} \ketbra{i+1 \pmod{d} }{i}  \quad \quad
	\sigma_z = \sum_{i=0}^{d-1} \omega_d^i \ketbra{i}{i}.
\end{align}
Moreover, if $U_x\sigma_xU_x^\dagger = \sigma_x^2$ and $U_z\sigma_zU_z^\dagger = \sigma_z^2$,
we can verify that 
\begin{align}
	U_xU_z = \1 = U_zU_x,
\end{align}
Hence, we define the new group by
\begin{equation}
\begin{aligned}
	\Pg =  \langle x, z, u, \J :  &zxz^{-1}x^{-1} = \J, [x,\J]=[z,\J]=[u,\J] = e, \\
	&uxu^{-1} = x^2, u^{-1}zu = z^2 \rangle. 
\end{aligned}
\end{equation}
\hl{Since $\Pg_d$'s relations satisfy the relations of $\Pg$, can we say $\Pg_d$ is a subgroup of $\Pg$?}
This group will be embedded in a solution group, $\Gamma_\Pg$, following Slofstra's embedding techniques.
See Appendix~\ref{sec:construct} for details.
\bibliographystyle{alphaurl}
\bibliography{quantum_correlation}
\appendix
%========================================
\section{Proof of \cref{thm:selftest} }
\label{sec:selftest}
%========================================
\begin{proof}
Following the techniques developed in Ref.~\cite{bamps2015}, the first step is to find a sum-of-square decomposition of 
\begin{align}
	\bar{\I}_\alpha = 2\sqrt{\alpha^2+1} \1 - \I_\alpha
	= \frac{2}{\sin(\mu)} \1 - \frac{\cos(\mu)}{\sin(\mu)}(A_0B_0+A_0B_1) -  A_1B_0 + A_1B_1.
\end{align} 
With the following notation
\begin{align*}
	Z_A = A_0 &\quad X_A = A_1\\
	Z_B = \frac{B_0+B_1}{2\cos(\mu)} &\quad X_B = \frac{B_0-B_1}{2\sin(\mu)},
\end{align*}
the two SOS decompositions that we use are
\begin{align}
	\label{eq:sos1}&\bar{\I}_\alpha = \frac{\sin(\mu)\bar{\I}_\alpha^2 + 4\sin(\mu)\cos(\mu)^2(Z_AX_B+X_AZ_B)^2}{4},\\
	\label{eq:sos2}&\bar{\I}_\alpha = \frac{\cos^2(\mu)}{\sin(\mu)}(Z_A-Z_B)^2 + \sin(\mu) (X_A-X_B)^2.
\end{align}
The verification is omitted here.

Suppose the quantum strategy $(\ket{\psi}, \{\tilde{A}_x\}_{x \in [2]}, \{\tilde{B}_{y in [2]}\}$ achieves that 
$\bra{\psi} \bar{\I}_\alpha \ket{\psi} \leq \epsilon$.
The second step is to establish bounds of the following form
\begin{align}
	&\|(\tilde{Z}_A-\tilde{Z}_B)\ket{\psi}\| \leq c_1 \sqrt{\epsilon}\\
	&\|(\tilde{X}_A(\1+\tilde{Z}_B)-\tilde{X}_B(\1-\tilde{Z}_A))\ket{\psi}\| \leq c_2 \sqrt{\epsilon}\\
	&\|(\tilde{X}_A-\tilde{X}_B)\ket{\psi}\| \leq c_3 \sqrt{\epsilon}\\
	&\|(\tilde{Z}_A\tilde{X}_A+\tilde{X}_A\tilde{Z}_A)\ket{\psi}\| \leq c_4 \sqrt{\epsilon}.
\end{align}
Now we write $s = \sin(\mu)$, $c = \cos(\mu)$ and define
\begin{align*}
	S_1 &= \frac{\sqrt{s}}{2} \bar{\I}_\alpha, \quad
	S_2 = \sqrt{s}c(\tilde{Z}_A\tilde{X}_B+\tilde{X}_A\tilde{Z}_B),\\
	S_3 &= \frac{c}{\sqrt{s}}(\tilde{Z}_A-\tilde{Z}_B),\quad
	S_4 = \sqrt{s}(\tilde{X}_A-\tilde{X}_B)
\end{align*}
then $\bar{\I}_\alpha = S_1^2 + S_2^2 = S_3^2 + S_4^2$ and $\bra{\psi}\bar{\I}_\alpha \ket{\psi} \leq \epsilon$ implies that 
$\bra{\psi}S^2_i \ket{\psi} \leq \epsilon$ and $\|S_i \ket{\psi} \| \leq \sqrt{\epsilon}$ for $i = 1,2,3,4$.
We can easily check that 
\begin{align*}
	c_1 = \frac{\sqrt{s}}{c}, \quad
	c_2 = \frac{1}{\sqrt{s}} + \frac{1}{c\sqrt{s}}, \quad
	c_3 = \frac{1}{\sqrt{s}}.
\end{align*}
where we use the relation that $\tilde{X}_A(\1+\tilde{Z}_B)-\tilde{X}_B(\1-\tilde{Z}_A) = S_4/s^{1/2} + S_2/(cs^{1/2})$.
To calculate $c_4$, we use the relation
\begin{align}
	\tilde{Z}_A\tilde{X}_A + \tilde{X}_A\tilde{Z}_A = \frac{S_2}{c\sqrt{s}} + \frac{\sqrt{s}\tilde{X}_AS_3}{c} + \frac{\tilde{Z}_AS_4}{\sqrt{s}}
\end{align}
and reach the conclusion that  
\begin{align}
	c_4 = \frac{1+c+s}{c\sqrt{s}}
\end{align}
where we use that fact that $\tilde{Z}_A, \tilde{X}_A$ are unitaries.

With appropriate substitutions, the rest of the proof follows the same derivation as that in Appendix A of Ref.~\cite{bamps2015},
so we omit it here.
\end{proof}
\end{document}
